When exploring the optimisation of mobile telecommunications, achieving a smooth transition of mobile connections between base stations, known as handover, is critical for maintaining uninterrupted and high-quality mobile service. While 5G is expected to provide ubiquitous indoor coverage in the coming years, a comprehensive analysis of indoor handover using real-world experimentation is not found in the literature. %Assuming that indoor handover will perform the same as outdoor settings is naive, as the indoor wireless channel is much more challenging.

This dissertation comprehensively examines indoor handover processes within LTE networks, utilising real-world experimental setups to explore the dynamics and challenges inherent to indoor signal transition. Utilising the findings from a series of experiments conducted within an LTE testbed designed to reenact the nuances of indoor environments, we analyse the efficacy of existing indoor handover algorithms and propose the need for more robust approaches.

This research's findings highlight the extent to which environmental factors impact handover efficacy. Substantial fluctuations in signal quality challenge the reliability of conventional handover mechanisms. Specifically, this dissertation found that increasing the hysteresis margin significantly reduced the rate of unnecessary handovers without adversely affecting the user experience regarding latency or throughput. 

Whilst the findings of this dissertation were somewhat limited by a lack of diversity within the utilised testing environments, they continue to highlight the requirement for future research into the specifics of this domain. This dissertation, therefore, concludes. This dissertation concludes with recommendations for network designers, operators, and future researchers, emphasizing the need for a nuanced approach to configuring handover parameters in indoor environments. 