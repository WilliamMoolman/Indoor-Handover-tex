\chapter{Introduction}

Over the last two decades, mobile operators have been focused on increasing bandwidth and reducing latency of mobile networks, as seen in the shift from 4G to 5G networks, to meet the increasing appetite of consumers to power technologies such as high-resolution streaming, video conferencing, virtual and augmented reality, and increasingly cloud-based AI processing \cite{cisco_cisco_2023}. Apart from the increasing capabilities of mobile networks, it is expected to extend the deployment of 5G networks to indoor environments to serve consumers' needs thoroughly. \textcolor{orange}{For instance, }This extension has already occurred in research facilities. However, public deployment is expected in the next decade \cite{zander_beyond_2016}.

\textcolor{orange}{I would suggest using User Equipment (UE) rather than transceiver. Transceiver is a generic term but the UE term is specific for cellular networks. Better use UE.}

There exists a gap in research on the performance of networks in indoor environments compared to outdoors, particularly looking at the handover process. This process attempts to connect each mobile cell to the best radio transceiver, the best of which is primarily determined through metrics such as Quality of Service (QoS) and Quality of Experience (QoE) \textcolor{orange}{it is not by RSPR?}. The handover process looks at switching to which radio transceiver a mobile cell is connected to when a better cell exists. This switching process, however, always incurs a throughput and latency penalty. In outdoor environments, the algorithms underpinning this handover process are well understood and tested \cite{mollel_survey_2021}. However, assuming such algorithms will perform similarly in indoor environments is naive.

Recent studies have examined the performance of 5G handovers, focusing on outdoor environments. The complexity of the indoor environment, influenced by much narrower propagation ranges and high signal loss due to Line of Sight (LOS) blockages and high radio absorption from obstacles and walls, presents unique challenges \cite{niknam_interference_2018}. These studies focus on testing classical algorithms and more state-of-the-art solutions, such as machine learning models. \cite{mollel_survey_2021}

Signal reflection, refraction, and absorption by walls and obstacles complicate indoor handover processes. These challenges necessitate sophisticated solutions that can dynamically adapt to changing indoor conditions. Therefore, we believe that algorithms developed for outdoor use will not necessarily translate to the indoor use case.

\textcolor{orange}{I am not sure if we can state the following: "contrast the performance to similar outdoor scenarios" since we have not taken outdoor measurements. If we state that, we need to cite some outdoor handover studies in the evaluation section and compare our results with theirs. Is something we can do? If so, please do it}

This paper aims to understand said handover process \textcolor{orange}{in indoor settings?}, understand \textcolor{orange}{(you wrote twice understand in a sentence, consider change by study)} the impact of a handover, contrast the performance to similar outdoor scenarios, and provide recommendations for possible algorithm improvements or adjustments to serve the indoor environment better and minimise drawbacks. To perform said analysis, real-world experiments must be performed, and the physical signals \textcolor{orange}{rather than physical signals, I would say capture Key Performance Indicator (KPIs) involved in handover like RSRP} must be captured for analysis to be reliable. For many papers exploring outdoor scenarios, simulations are used instead of real-world setups due to the complexity of maintaining a radio network, as well as the relatively high costs of equipment. Later in the paper, we face similar roadblocks and document our various steps.

\textcolor{orange}{BTW, do we need to use UK or US English? Analyze or analyse?}

\textbf{Clarified Objectives:} This research aims to:
\begin{itemize}
    \item \textcolor{orange}{Propose a testbed to study handover to be used to analyse handover?}
    \item Analyze the performance of existing handover algorithms in indoor environments.
    \item Propose adjustments or new algorithms designed to enhance Quality of Service (QoS) and Quality of Experience (QoE) in indoor 5G networks. \textcolor{orange}{you have already defined the acronym Quality of Service (QoS), consider just using QoS, same for QoE }
\end{itemize}

\textcolor{orange}{You wrote a long sentence in the below paragraph, more than 4 lines. I suggest writing there-lines sentences at most}

The study will conduct real-world experiments in various indoor settings to collect signal data \textcolor{orange}{KPIs?} to achieve these objectives. Simulations can be used in outdoor environments, as modelling the radio characteristics in a free air space is much easier, using techniques such as propagation loss algorithms such as \textcolor{orange}{twice such as in a sentence, change it by like} shadow-log loss \cite{sun_path_2015} compared to modelling the radio absorption and reflection of walls, obstacles and dynamic objects such as human movement, which will deviate significantly from reality. Furthermore, studies that demonstrate the use of such simulations inherently prioritise the outdoor accuracy of said simulations. \textcolor{orange}{this sentence is a strong statement, could we cite a paper to validate that statement?}

Understanding indoor handovers is critical as it affects user experience significantly, especially with the increased demand for high-bandwidth applications. This paper aims to explore the impact of handovers on throughput and connectivity, as well as the frequency of handovers, and to determine whether all occurring handovers are necessary. \textcolor{orange}{Seems that this paragraph is quite similar to the one that is above. The one that starts with "This paper aims to understand said handover process", might we combine both?} 

This paper's findings highlight the need for more careful setup and handling of indoor handovers and pave the way for more extensive real-world research. They also pave the way \textcolor{orange}{twice pave the way} for subsequent exploratory studies that analyse other indoor cellular network performance aspects. Ultimately, this work aims to enhance connectivity and service reliability in indoor environments, addressing the evolving needs of mobile network users. \todo{Insert results once ready} \textcolor{orange}{we need a paragraph to explain results.}

\textcolor{orange}{Consider writing this: This paper significantly contributes. We can state that since we have already done it by explaining our findings.}

In conclusion, this paper aims to significantly contribute to understanding and improving indoor handover processes in 5G networks, \textcolor{orange}{by?} providing a practical foundation and motivation for further research on development in this unexplored environment \textcolor{orange}{unexplored field rather than environment}. 



% Currently at 2 pages, attempt to expand to 3-4
