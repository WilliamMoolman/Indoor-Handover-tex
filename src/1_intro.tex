\chapter{Introduction}

Over the past two decades, mobile operators have predominantly focused on increasing bandwidth and reducing the latency of mobile networks. This is reflected in the shift from 4G to 5G networks, which was implemented to meet the increasing consumer appetite for high-bandwidth technologies such as high-resolution streaming, video conferencing, virtual and augmented reality, and increasingly cloud-based AI processing \cite{cisco_cisco_2023}. To date, this 5G transition has been focused on improving the capabilities of existing mobile networks; however, as consumer demand grows, it is expected that 5G networks will begin to be deployed within indoor environments. Thus far, indoor deployment has been limited to research facilities, however public deployment is expected in the next decade \cite{zander_beyond_2016}. 

Whilst extensive research has been performed on the performance of mobile networks in outdoor environments, there is currently a gap in the literature when exploring the performance of mobile networks within indoor environments, particularly when analysing the handover process. The handover process attempts to connect each mobile cell to the optimum radio base station (eNB), primarily determined by the signal strength measured by RSRP or RSRQ. By switching a mobile cell once its connection degrades, the handover process maintains user experience as per Quality of Service (QoS) and Quality of Experience (QoE) metrics; however, a throughput and latency penalty is simultaneously incurred.  
% 
% In outdoor environments, the algorithms underpinning this handover process are well understood and tested using classical algorithms and state-of-the-art solutions, such as machine learning models \cite{mollel_survey_2021}. However, such algorithms poorly translate to indoor environments; the significantly narrower propagation ranges, high rates of Line of Sight (LOS) signal loss, and higher radio absorption from walls and obstacles create unique challenges \cite{niknam_interference_2018}. The arising signal reflection, refraction, and absorption complicate indoor handover processes. This dissertation, therefore, proposes that these challenges necessitate sophisticated solutions that can dynamically adapt to changing indoor conditions.  

% This paper aims to understand the handover process in indoor settings, study the impact of a handover, contrast the performance to results found in literature, and provide recommendations for possible algorithm improvements or adjustments to serve the indoor environment better and minimise drawbacks. Real-world experiments must be performed to understand this process, and network operating data, such as RSRP, throughput and latency, must be captured for the analysis to be reliable. For much of the literature, simulations are used instead of real-world setups due to the complexity of maintaining a radio network, as well as the relatively high costs of equipment. This paper will initially utilise simulations but will use real-world experiments once the limitations inherent to simulations are reached.

% The understanding and calibration of the handover process is critical as it affects user experience significantly, especially with the increased demand for high-bandwidth applications. This paper aims to explore the impact of handovers on throughput and connectivity, as well as the frequency of handovers, and to determine whether all occurring handovers are necessary.

% \textbf{Clarified Objectives:} This research aims to:
% \begin{itemize}
%     \item Propose a testbed to study handover to be used to analyse handover
%     \item Analyse the performance of existing handover algorithms in indoor environments.
%     \item Propose adjustments or new algorithms designed to enhance the user experience in indoor 5G networks.
% \end{itemize}

% The study will conduct simulations alongside real-world experiments in various indoor settings to collect network usage data to achieve these objectives. Simulations can be used in outdoor environments, as modelling the radio characteristics in a free-air space is straightforward using propagation loss algorithms such as log-distance log loss \cite{sun_path_2015}. Indoor environments require modelling the radio absorption and reflection of walls, obstacles and dynamic objects such as human movement, which will deviate significantly from reality.

% This paper's findings highlight the need for more careful setup and handling of indoor handovers and layout recommendations for more extensive real-world research. They also pave the way for subsequent exploratory studies that analyse other aspects of indoor cellular network performance. Ultimately, this work aims to enhance connectivity and service reliability in indoor environments, addressing the evolving needs of mobile network users.

% Through a series of real-world experiments and simulations, we have generated insights into the predictability of the hysteresis parameter, the limitations of existing simulation models for indoor scenarios, and the relationship between hysteresis values and ping-pong rates. These findings show the inherent instability of indoor signals and demonstrate environmental factors' considerable impact on network performance quality and reliability. Furthermore, our work examines the effects of handovers on network resilience, proposing a minimisation of handover rates to improve network robustness. However, acknowledging the limitations of our current methodologies, we present future research avenues for inquiry that span more complex simulation frameworks, diverse testing environments, and the exploration of predictive handover strategies using machine learning.

% In conclusion, this paper significantly contributes to understanding and improving handover processes in indoor networks by providing a practical foundation and motivation for further research on development in this unexplored field to prepare for the upcoming 5G deployments.

The aims of this dissertation include:  
\begin{enumerate}
    \item To propose and deploy a testbed to study handover to be used to analyse handover.
    \item To better understand the handover process within indoor settings and quantify the impact of a handover on user throughput and latency.
    \item To propose adjustments or new algorithms to enhance the user experience in indoor 5G networks.
\end{enumerate}
A robust testbed is essential to conducting a reliable analysis of the handover process, forming our initial aim. This paper will conduct simulations alongside real-world experiments in various indoor settings to collect network usage data to better achieve the proceeding aims. For much of the literature, simulations are used instead of real-world setups due to the complexity of maintaining a radio network, as well as the relatively high costs of equipment. This paper will initially utilise simulations but will use real-world experiments once the limitations inherent to simulations are reached. Simulations can be used in outdoor environments, as modelling the radio characteristics in a free-air space is straightforward using propagation loss algorithms such as log-distance log loss \cite{sun_path_2015}. Indoor environments require modelling the radio absorption and reflection of walls, obstacles and dynamic objects such as human movement, which will deviate significantly from reality. However, a testbed using real-world hardware is essential, during which network operating data, such as RSRP, throughput and latency, must be captured for the analysis to be reliable. 

Our second aim for the paper is the crux of the paper, as the understanding and calibration of the handover process is critical as it affects user experience significantly, especially with the increased demand for high-bandwidth applications. This paper aims to explore the impact of handovers on throughput and connectivity, as well as the frequency of handovers, and to determine whether all occurring handovers are necessary. 

Our final aim is to use the insights gathered from our experimentation to propose adjustments to existing handover algorithms, suggest possible new techniques, and highlight how network design influences the handover process. 

This dissertation’s findings highlight the need for more careful setup and handling of indoor handovers and layout recommendations for more extensive real-world research. They also pave the way for subsequent exploratory studies that analyse other aspects of indoor cellular network performance. Ultimately, this work aims to enhance connectivity and service reliability in indoor environments, addressing the evolving needs of mobile network users. Through simulation, we examined the limitations of existing simulation models for indoor scenarios and confirmed the relationship between the hysteresis parameter and ping-pong rates. Through real-world experimentation, we have shown the inherent instability of indoor signals and demonstrated the considerable impact of environmental factors on network performance and reliability. Furthermore, our work examines the effects of handovers on network resilience, proposing a minimisation of handover rates to improve network robustness. However, acknowledging the limitations of our current methodologies, we present future research avenues for inquiry that span more complex simulation frameworks, diverse testing environments, and the exploration of predictive handover strategies using machine learning. 

In conclusion, this paper significantly contributes to understanding and improving handover processes in indoor networks by providing a practical foundation and motivation for further research on development in this unexplored field to prepare for the upcoming 5G deployments. 