\chapter{Conclusion}
Reflecting on the ambitious research in this paper, we aimed to develop a comprehensive understanding of the handover processes of 5G networks inside relatively unexplored indoor environments. This exploration was motivated by the increasing demand for seamless connectivity to support high-bandwidth applications, from high-resolution streaming to cloud-based AI processing. Through a series of rigorous real-world experiments and simulations, this paper has uncovered the complexities and difficulties of indoor deployment and handover and highlighted several future research pathways for enhancing network performance and user experience. In this chapter, we highlight the key finding of our research, their practical importance, and the avenues they open for future exploration.

\section{Synthesis of Key Insights}
\begin{itemize}
    \item We validated the operational integrity of srsRAN within LTE standards, highlighting the predictability of the hysteresis parameter in the handover process.
    \item We delineated the limitations of current simulation models in mirroring the intricacies of indoor settings, focusing our attention on real-world testbeds for more granular insights.
    \item We identified a critical correlation between hysteresis settings and ping-pong rates, advocating for tailored adjustments to optimize handover algorithms for indoor use.
    \item On experimentation, we uncovered the inherent instability of indoor signals, emphasizing the significant role environmental factors play in handover dynamics and network stability.
    \item We illustrated through latency and throughput assessments the tangible impacts of handovers on network performance and recommended minimizing handover frequency in pursuit of network resilience.
\end{itemize}

\section{Reflections on Network Design and Management}
\begin{itemize}
    \item We illustrated the necessity for nuanced application of handover parameters, championing higher hysteresis values and restrained cell density to mitigate the challenges posed by high signal fluctuation and excessive indoor handovers.
    \item We highlighted the critical need for fortifying network robustness against handover-induced disruptions, potentially through innovative handover algorithms or infrastructural enhancements.
\end{itemize}

\section{Limitations and Horizons for Future Inquiry}
\begin{itemize}
    \item We acknowledged the constraints of simulation models and the controlled nature of our testbed experiments, setting the stage for developing more sophisticated simulation frameworks and diversifying testing environments.
    \item We proposed a series of expansive studies in various indoor settings to glean a broader understanding of network behaviour, signalling the need to test environments that mimic the complexity and variability of real-world use cases.
    \item We further suggested the examination of 5G network dynamics in indoor scenarios, considering the distinct radio properties of 5G networks and their potential to unveil new insights relative to LTE systems.
    \item Finally, we advocated for exploring predictive handover strategies, leveraging machine learning to anticipate and mitigate potential service degradation before it compromises user experience.
\end{itemize}

\section{Concluding Reflections}
In alignment with the objectives delineated at the outset of this paper, our findings underscore the paramount importance of rethinking handover processes for the indoor deployment of 5G networks. By navigating the labyrinth of challenges unique to indoor environments, from signal reflections to LOS blockages, this research contributes substantively to our understanding of indoor network performance. Furthermore, it lays the groundwork for more nuanced algorithm development and network management strategies that promise to elevate the quality of service and experience for users in indoor settings. As we stand on the precipice of widespread public deployment of 5G networks in indoor environments, the insights derived from this study serve as both a beacon and a foundation for the ongoing quest to optimize connectivity and service reliability in the face of evolving consumer demands and technological landscapes.
