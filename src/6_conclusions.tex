\chapter{Conclusion}
Reflecting on the research in this paper, we aimed to develop a comprehensive understanding of the handover processes of 5G networks inside relatively unexplored indoor environments. This exploration was motivated by the increasing demand for seamless connectivity to support high-bandwidth applications, from high-resolution streaming to cloud-based AI processing. Through a series of real-world experiments and simulations, this paper has uncovered the complexities and difficulties of indoor deployment and handover and highlighted several future research pathways for enhancing network performance and user experience. In this chapter, we highlight the key finding of our research, their practical importance, and the avenues they open for future exploration.

\section{Synthesis of Key Insights}
\begin{itemize}
    \item We validated the operational integrity of srsRAN within LTE standards, highlighting the predictability of the hysteresis parameter in the handover process.
    \item We highlighted the limitations of current simulation models in emulating indoor settings, focusing our attention on real-world testbeds for more granular insights.
    \item We identified a strong correlation between hysteresis values and ping-pong rates, and highlighted the need to optimize handover algorithms for indoor use.
    \item On experimentation, we encountered the inherent instability of indoor signals. This demonstrated the significant effect of environmental factors on handover rates and network stability.
    \item We quantified the latency and throughput impacts of handovers on network performance. This led to the recommendation of minimising handover frequency for a more resilient network.
\end{itemize}

\section{Reflections on Network Design and Management}
\begin{itemize}
    \item We encounter the need for careful handover management, higher hysteresis values and reduced cell density to mitigate the difficulties posed by high signal fluctuation and excessive indoor handovers.
    \item We highlighted the need for increased network robustness against handover-induced disruptions, potentially through new handover algorithms or infrastructural enhancements.
\end{itemize}

\section{Limitations and Horizons for Future Inquiry}
\begin{itemize}
    \item We acknowledged the constraints of simulation models and the controlled nature of our testbed experiments, laying the foundations for developing more sophisticated simulation frameworks and diversifying testing environments.
    \item We proposed a series of further indoor settings to extend this study to obtain a broader understanding of network behaviour, signalling the need to test environments that mimic the complexity and variability of real-world use cases.
    \item We further suggested the extension to using 5G transmission in our testbed, to better consider the distinct radio properties they have.
    \item Finally, we considered the usage of predictive handover strategies, leveraging machine learning to anticipate and mitigate potential service degradation before it compromises user experience.
\end{itemize}

\section{Concluding Reflections}
In alignment with the objectives at the outset of this paper, our findings underscore the importance of rethinking handover processes for the indoor deployment of 5G networks. By exploring the difficulties unique to indoor environments, from signal reflections to LOS blockages, this research challenges the naive understanding of indoor network performance. Furthermore, it lays the groundwork for more extensive, large-scale testing of the indoor environment to better develop strategies to improve the end-user experience.% As we are at the beginning of the widespread public deployment of 5G networks in indoor environments, the insights found in this study serve as both a beacon and a foundation for the ongoing quest to optimize connectivity and service reliability in the face of evolving consumer demands and technological landscapes.
