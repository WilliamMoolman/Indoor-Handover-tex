\chapter{Background}
\section{Handover}
\begin{itemize}
  \item Handover in wireless communication networks
  \item Ensures continuous connectivity for mobile devices
  \item Involves the transfer of connections from one cell to another
  \item Integral part of mobility management in cellular networks
  \item Inter vs Intra frequency, inter vs intra RAT
  \item S1, X2, Xn
\end{itemize}
\begin{itemize}
  \item 4G vs 5G - focus on 4G
  \item Fundamentals of radio communication in the context of handover
  \item Challenges associated with Next Generation Radio, especially in 5G networks. Factors like mmWave radio, LOS dependency, high path loss, microcells, indoor environments, and HetNets
  \item Key concepts: signal strength (RSRP) and propagation loss
  \item Challenges in handover management and optimization
\end{itemize}


\section{Mobility Management + Load Balancing}
\begin{itemize}
\item[Mobility Management]
  \item Key role in cellular networks for seamless connectivity
  \item Management of mobile user movement within the network
  \item Optimization to ensure quality of service (QoS)
  \item Challenges in handling fast-moving users and handovers
  \item Vital for maintaining Quality of Service (QoS)
  \item Efficient handover strategies as part of mobility management
  \item Importance in load balancing and network resource allocation
\item[Load Balancing]
    \item Essential for maintaining network performance in cellular systems
    \item Equitable distribution of traffic among base stations
    \item Ensures efficient resource utilization and reduces congestion
    \item Key in improving network reliability and user satisfaction
    \item Load balancing algorithms for dynamic adaptation
    \item Strategies in over coming these mentioned in Section \textbf{ALGORITHMS}
    \item[{hatipoglu_handover-based_2020}] The paper presented a handover-based load balancing algorithm for HetNets. The paper utilised UE speeds in determining handover, an metric which is unrealistic to obtain in practice. The algorithm itself is fairly simplistic, and while the paper showed good results, the algorithm does not consider key metrics such as RSRP, or connection speed.
\end{itemize}


\section{Handover Algorithms}
\subsection{Classical Approach}
\begin{itemize}
    \item Hysteresis
    \item Time to Trigger
\end{itemize}
\subsection{Alternative Heuristics}
\begin{itemize}
    \item UE Speed and more
\end{itemize}
\subsection{Machine Learning}
\begin{itemize}
    \item Categorised into network-based and optical-assisted
\end{itemize}
\subsubsection{Network Based}
\begin{itemize}
    \item Supervised
    \item Unsupervised
    \item Reinforcement
    \setlength{\itemindent}{2em}
    \item Q-learning
    \item[{yajnanarayana_5g_2020}] The paper presented a handover algorithm using Contextual Multi-Armed Bandit reinforcement learning. This ML model provided modest improvements (0.3dB) in average RSRP of devices. The authors also wrote their own network simulator, one which used sophisticated propagation models, such as log-shadowing, as well as the WINNER UMa Model. The code used to simulate it was not released however, providing very little reproducability of the paper. The improvements were also not very large, and better results could possibly be obtained with a deep learning based Reinforcement model
    \item Deep Learning
\end{itemize}
\subsubsection{Optical Based}
\begin{itemize}
    \item Tracking UEs using cameras
    \item Predicting LOS blockages
\end{itemize}
\section{Handover Testbeds}
\begin{itemize}
  \item Crucial for evaluating and testing network performance
  \item Create realistic environments for experiments
  \item srsRAN, OpenAirInterface, and Aether are widely used simulators
  \item I have chosen srsRAN: Better code quality for research, solid tutorials, and documentation - 4G only
  \item Importance of realistic simulations in research and development
    \item[{powell_handover_2021}] The paper presented a testing framework for 4G experiments using srsRAN, and performed basic experiments with it. The framework was presented clearly, and I was able to replicate the results of their simulation. The experiments they performed however were limited as signal strength was arbitrarily introduced by attenuating a signal, and did not utilise signal propogation algorithms.
\end{itemize}


\section{Research Gap}
\begin{itemize}
    \item Load Balancing in realistic testbed
    \item Improve LB with better heuristics
\end{itemize}