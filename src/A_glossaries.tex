\chapter{Glossary}
\newcommand{\acronym}[2]{\newacronym[sort=#1]{#1}{#1}{#2}}

\textcolor{orange}{eNB refers to LTE which surprisingly 4G and LTE are not the same protocol \url{https://www.uctel.co.uk/blog/4g-vs-lte-understanding-the-difference-between-4g-and-lte}. enB means evolved Node B (eNB)- LTE base station.
next-generation Node B (gNB)
User Equipment (UE)}

\acronym{eNB}{A 4G base station, a radio transceiver that provides the radio connection to the UE}
\acronym{UE}{Any mobile device that can connect to the mobile network as a client}
\acronym{LOS}{Line of Sight}
\acronym{gNB}{A 5G base station}
\acronym{ML}{Machine Learning}
\acronym{HOPP}{Handover Ping Pong}
\acronym{HOF}{Handover Failure}
\acronym{HO}{Handover}
\acronym{RSRP}{Reference Signal Receiver Power - a measure of signal quality}
\acronym{SINR}{Signal-to-Interference-plus-Noise Ratio}
\acronym{RSRQ}{Reference Signal Receiver Quality - a measure of signal quality}
\acronym{CQI}{Channel Quality Indicator}
\acronym{PCI}{Physical Cell Index - a value representing a physical cell}

\glsaddall
\renewcommand{\glossarysection}[2][]{}
\printglossary[type=\acronymtype]